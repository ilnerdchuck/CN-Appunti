\section{Segnali Analogici}
    \subsection{Grandezze dei segnali Analogici}
    
        \subsubsection{Potenza istantanea}
            \[
                P_{x} \triangleq |x_{(t)}|^2   
            \]
        \subsubsection{Energia}
            \[
                E_{x} \triangleq \int_{-\infty}^{\infty} P_{x}(t) \,dt = \int_{-\infty}^{\infty} |x_{(t)}|^2 \,dt    
            \]
        \subsubsection{Potenza Media}
            Definiamo il \index{Segnale Troncato}{\bf Segnale Troncato}:
                \[
                    x_{(t)} = X_{(t)} \triangleq 
                    \begin{cases}
                        x_{(t)} \hspace{1cm} -\frac{T}{2} \leq t \leq \frac{T}{2} \\
                        0 \hspace{2cm}altrove
                    \end{cases}
                    \]  
                    \begin{center}
                        \em T = Periodo di osservazione
                    \end{center}
                \begin{figure}[h]
                    \centering
                    \includegraphics[width=4cm]{media/uwu.png}
                    \caption{Segnale troncato}
                    \label{fig:troncato}
                \end{figure}
            

            % NON CAPISCO COSA É POTENZA MEDIA E COSA SIA POTENZA ISTANTANEA CHE CAVOLO DI RELAZIONE
            % USO PER PASSARE DA PxT A Px  
            La potenza media é:
            \[
                P_{x_{T}} \triangleq E_{x_{T}}   
            \]
            dalla quale possiamo ricavare la potenza istantanea se $T \rightarrow \infty \Rightarrow P_{x_{T}} = P_{x}$:
            \[
                P_{x} \triangleq \lim_{T\rightarrow\infty} \frac{E_{x_{T}}}{T} =\lim_{T\rightarrow\infty} \frac{1}{T} \int_{-\frac{T}{2}}^{\frac{T}{2}}  |x_{(t)}|^2 \,dt    
            \]  
        \subsubsection{Valore Efficace}
        
        \subsubsection{Valore Medio}

    \subsection{Analisi energetiche su segnali comuni}
        \subsubsection{Costante}

        \subsubsection{Sinusoide}
        
        \subsubsection{Gradino}
        
        \subsubsection{Rettangolo}
        
        \subsubsection{Esponenziale unilatera}
        
        \subsubsection{Esponenziale bilatera}
        
        \subsubsection{segno $\mathbf{sgn(x_{(t)})}$}