\section{Segnali Analogici}
    \subsection{Grandezze dei segnali Analogici}
    
        \subsubsection{Potenza istantanea}
            \[
                P_{x} \triangleq |x_{(t)}|^2   
            \]
        \subsubsection{Energia}
            \[
                E_{x} \triangleq \int_{-\infty}^{\infty} P_{x}(t) \,dt = \int_{-\infty}^{\infty} |x_{(t)}|^2 \,dt    
            \]
        \subsubsection{Potenza Media}
            Definiamo il \index{Segnale Troncato}{\bf Segnale Troncato}:
                \[
                    x_{(t)} = X_{(t)} \triangleq 
                    \begin{cases}
                        x_{(t)} \hspace{1cm} -\frac{T}{2} \leq t \leq \frac{T}{2} \\
                        0 \hspace{2cm}altrove
                    \end{cases}
                    \]  
                    \begin{center}
                        \em T = Periodo di osservazione
                    \end{center}
                \begin{figure}[h]
                    \centering
                    \includegraphics[width=4cm]{media/uwu.png}
                    \caption{Segnale troncato}
                    \label{fig:troncato}
                \end{figure}
            

            % NON CAPISCO COSA É POTENZA MEDIA E COSA SIA POTENZA ISTANTANEA CHE CAVOLO DI RELAZIONE
            % USO PER PASSARE DA PxT A Px  
            La potenza media é:
            \[
                P_{x_{T}} \triangleq \frac{E_{x_{T}}}{T}    
            \]
            dalla quale possiamo ricavare se $T \rightarrow \infty \Rightarrow P_{x_{T}} = P_{x}$:
            \[
                P_{x} \triangleq \lim_{T\rightarrow\infty} \frac{E_{x_{T}}}{T} =\lim_{T\rightarrow\infty} \frac{1}{T} \int_{-\frac{T}{2}}^{\frac{T}{2}}  |x_{(t)}|^2 \,dt    
            \]  
        \subsubsection{Valore Efficace}
                \[    
                    x_{eff} \triangleq \sqrt{P_{x}}
                \]
        
        \subsubsection{Valore Medio}

                % TO DO: rivedi qui per evitare un page break
                    \[
                        x_{m} \triangleq \lim_{T\rightarrow\infty} \frac{1}{T} \int_{-\infty}^{\infty}  x_{(t)_T} \,dt = \lim_{T\rightarrow\infty} \frac{1}{T} \int_{-\frac{T}{2}}^{\frac{T}{2}}  x_{(t)} \,dt 
                    \]
                    \[
                        x_{(t)_T}\ =\ Segnale\ troncato
                    \]
                    
    \subsection{Analisi energetiche su segnali comuni}
        \subsubsection{Costante}
            $x_{(t)} = A\ \ \forall t$
            \begin{figure}[h]
                \centering
                \includegraphics[width=4cm]{media/uwu.png}
                \caption{Segnale costante}
                \label{fig:segnale costante}
            \end{figure}
            \begin{itemize}
                \item {Energia:
                        \[
                            E_{x} = \int_{-\infty}^{\infty} P_{x}(t) \,dt = \int_{-\infty}^{\infty} |x_{(t)}|^2 \,dt = \int_{-\infty}^{\infty} A^2 \,dt = \infty 
                        \]
                }
                \item {Potenza Media:
                        \[
                            P_{x} = \lim_{T\rightarrow\infty} \frac{E_{x_{T}}}{T} = \lim_{T\rightarrow\infty} \frac{1}{T} \int_{-\frac{T}{2}}^{\frac{T}{2}}  |x_{(t)}|^2 \,dt = \lim_{T\rightarrow\infty} \frac{1}{T} \int_{-\frac{T}{2}}^{\frac{T}{2}} A^2 \,dt = A^2     
                        \]
                }
                \item {Valore Efficace:
                        \[
                            x_{eff} = \sqrt{P_{x}} = \sqrt{A^2} = |A|
                        \]
                }
                \item {Valore Medio:
                        \[
                            x_{m} = \lim_{T\rightarrow\infty} \frac{1}{T} \int_{-\frac{T}{2}}^{\frac{T}{2}}  x_{(t)} \,dt = \lim_{T\rightarrow\infty} \frac{1}{T} \int_{-\frac{T}{2}}^{\frac{T}{2}}  A \,dt = \lim_{T\rightarrow\infty} \frac{1}{T} AT = A 
                        \]
                }
            \end{itemize}
        
        \subsubsection{Sinusoide}
            $x_{(t)} = A \cos(2 \pi f_0 t +\phi)$
            \begin{figure}[H]
                \centering
                \includegraphics[width=4cm]{media/uwu.png}
                \caption{Segnale sinusoidale}
                \label{fig:segnale sinusoidale}
            \end{figure}
            
            \begin{itemize}
                \item {Energia:
                    \[
                        E_{x} = \int_{-\infty}^{\infty} |x_{(t)}|^2 \ dt = \int_{-\infty}^{\infty} A^2 \cos^2(2\pi f_0 t + \phi) dt 
                    \]
                    Ricaviamo dalla $(1)$ \ref{Trigonometria} il $\sin^2(\alpha)$ e $(2.1)$ \ref{Trigonometria_Duplicazione} $\cos(2\alpha) = \frac{1+ \cos^2(\alpha)}{2}$
                    \[
                        E_{x} = A^2  \int_{-\infty}^{\infty} \frac{1}{2} + \frac{\cos(4\pi f_0 t + 2\phi)}{2} dt \\
                    \]
                }
                \item {Potenza Media:
                }
                \item {Valore Efficace:
                }
                \item {Valore Medio:
                }
            \end{itemize}

            %Richimo per il label del formulario $(1)$ e $(2)$ \ref{Trigonometria} 

        \subsubsection{Gradino}
        $x_{(t)} = A \cos(2 \pi f_0 t +\phi)$
        \begin{figure}[H]
            \centering
            \includegraphics[width=4cm]{media/uwu.png}
            \caption{Segnale gradino}
            \label{fig:segnale gradino}
        \end{figure}        
        
        \subsubsection{Rettangolo}
        $x_{(t)} = A \cos(2 \pi f_0 t +\phi)$
        \begin{figure}[H]
            \centering
            \includegraphics[width=4cm]{media/uwu.png}
            \caption{Segnale rettangolo}
            \label{fig:segnale rettangolo}
        \end{figure}        
        
        \subsubsection{Esponenziale unilatera}
        $x_{(t)} = A \cos(2 \pi f_0 t +\phi)$
        \begin{figure}[H]
            \centering
            \includegraphics[width=4cm]{media/uwu.png}
            \caption{Segnale esponenziale unilatera}
            \label{fig:segnale esponenziale unilatera}
        \end{figure}        
        
        \subsubsection{Esponenziale bilatera}
        $x_{(t)} = A \cos(2 \pi f_0 t +\phi)$
        \begin{figure}[H]
            \centering
            \includegraphics[width=4cm]{media/uwu.png}
            \caption{Segnale esponenziale bilatera}
            \label{fig:segnale esponenziale bilatera}
        \end{figure}        
        
        \subsubsection{segno $\mathbf{sgn(x_{(t)})}$}
        $x_{(t)} = A \cos(2 \pi f_0 t +\phi)$
        \begin{figure}[H]
            \centering
            \includegraphics[width=4cm]{media/uwu.png}
            \caption{Segnale sgn(x)}
            \label{fig:segnale sgn(x)}
        \end{figure}