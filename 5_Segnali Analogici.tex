\section{Segnali Analogici}
    \subsection{Grandezze dei segnali Analogici}
    
        \subsubsection{Potenza istantanea}\label{Potenza istantanea}
            \[
                P_{x} \triangleq |x_{(t)}|^2   
            \]
        \subsubsection{Energia}
            \[
                E_{x} \triangleq \int_{-\infty}^{\infty} P_{x}(t) \,dt = \int_{-\infty}^{\infty} |x_{(t)}|^2 \,dt    
            \]
        \subsubsection{Potenza Media}\label{Potenza media}
            Definiamo il \index{Segnale Troncato}{\bf Segnale Troncato}:
                \[
                    x_{(t)} = X_{(t)} \triangleq 
                    \begin{cases}
                        x_{(t)} \hspace{1cm} -\frac{T}{2} \leq t \leq \frac{T}{2} \\
                        0 \hspace{2cm}altrove
                    \end{cases}
                    \]  
                    \begin{center}
                        \em T = Periodo di osservazione
                    \end{center}
                \begin{figure}[h]
                    \centering
                    \includegraphics[width=4cm]{media/uwu.png}
                    \caption{Segnale troncato}
                    \label{fig:troncato}
                \end{figure}
            

            % NON CAPISCO COSA É POTENZA MEDIA E COSA SIA POTENZA ISTANTANEA CHE CAVOLO DI RELAZIONE
            % USO PER PASSARE DA PxT A Px  
            La potenza media é:
            \[
                P_{x_{T}} \triangleq \frac{E_{x_{T}}}{T}    
            \]
            dalla quale possiamo ricavare se $T \rightarrow \infty \Rightarrow P_{x_{T}} = P_{x}$:
            \[
                P_{x} \triangleq \lim_{T\rightarrow\infty} \frac{E_{x_{T}}}{T} =\lim_{T\rightarrow\infty} \frac{1}{T} \int_{-\frac{T}{2}}^{\frac{T}{2}}  |x_{(t)}|^2 \,dt    
            \]  
        \subsubsection{Valore Efficace}\label{Valore Efficace}
                \[    
                    x_{eff} \triangleq \sqrt{P_{x}}
                \]
        
        \subsubsection{Valore Medio}\label{Valore medio}

                % TO DO: rivedi qui per evitare un page break
                    \[
                        x_{m} \triangleq \lim_{T\rightarrow\infty} \frac{1}{T} \int_{-\infty}^{\infty}  x_{(t)_T} \,dt = \lim_{T\rightarrow\infty} \frac{1}{T} \int_{-\frac{T}{2}}^{\frac{T}{2}}  x_{(t)} \,dt 
                    \]
                    \[
                        x_{(t)_T}\ =\ Segnale\ troncato
                    \]
                    
    \subsection{Analisi energetiche su segnali comuni}
        \subsubsection{Costante}
            $x_{(t)} = A\ \ \forall t$
            \begin{figure}[h]
                \centering
                \includegraphics[width=4cm]{media/uwu.png}
                \caption{Segnale costante}
                \label{fig:segnale costante}
            \end{figure}
            \begin{itemize}
                \item {Energia:
                        \[
                            E_{x} = \int_{-\infty}^{\infty} P_{x}(t) \,dt = \int_{-\infty}^{\infty} |x_{(t)}|^2 \,dt = \int_{-\infty}^{\infty} A^2 \,dt = \infty 
                        \]
                }
                \item {Potenza Media:
                        \[
                            P_{x} = \lim_{T\rightarrow\infty} \frac{E_{x_{T}}}{T} = \lim_{T\rightarrow\infty} \frac{1}{T} \int_{-\frac{T}{2}}^{\frac{T}{2}}  |x_{(t)}|^2 \,dt = \lim_{T\rightarrow\infty} \frac{1}{T} \int_{-\frac{T}{2}}^{\frac{T}{2}} A^2 \,dt = A^2     
                        \]
                }
                \item {Valore Efficace:
                        \[
                            x_{eff} = \sqrt{P_{x}} = \sqrt{A^2} = |A|
                        \]
                }
                \item {Valore Medio:
                        \[
                            x_{m} = \lim_{T\rightarrow\infty} \frac{1}{T} \int_{-\frac{T}{2}}^{\frac{T}{2}}  x_{(t)} \,dt = \lim_{T\rightarrow\infty} \frac{1}{T} \int_{-\frac{T}{2}}^{\frac{T}{2}}  A \,dt = \lim_{T\rightarrow\infty} \frac{1}{T} AT = A 
                        \]
                }
            \end{itemize}
        
        \subsubsection{Sinusoide}
            $x_{(t)} = A \cos(2 \pi f_0 t +\phi)$
            \begin{figure}[H]
                \centering
                \includegraphics[width=4cm]{media/uwu.png}
                \caption{Segnale sinusoidale}
                \label{fig:segnale sinusoidale}
            \end{figure}
            
            \begin{itemize}
                \item {Energia:
                    \[
                        E_{x} = \int_{-\infty}^{\infty} |x_{(t)}|^2 \ dt = \int_{-\infty}^{\infty} A^2 \cos^2(2\pi f_0 t + \phi) dt 
                    \]
                    Ricaviamo dalla $(1)$ \ref{Trigonometria} il $\sin^2(\alpha)$ e lo sostituiamo $(2.1)$ \ref{Trigonometria_Duplicazione} \\ $\cos(2\alpha) = \frac{1+ \cos^2(\alpha)}{2}$
                    \begin{align}
                            &= A^2  \int_{-\infty}^{\infty} \frac{1}{2} + \frac{\cos(4\pi f_0 t + 2\phi)}{2} dt \nonumber \\
                            &= A^2  \int_{-\infty}^{\infty} \frac{1}{2} dt + A^2 \int_{-\infty}^{\infty} \frac{\cos(4\pi f_0 t + 2\phi)}{2} dt \nonumber \\
                            &= \infty +\eval{\frac{A}{2} \frac{1}{4\pi f_0} \sin(4\pi f_0 t) }_{-\infty}^{\infty} = \infty \nonumber
                    \end{align}
                }
                \item {Potenza Media:
                    \begin{align}
                        P_{x} &=\lim_{T\rightarrow\infty}  \frac{1}{T} \int_{-\frac{T}{2}}^{\frac{T}{2}}  |x_{(t)}|^2 \,dt =\lim_{T\rightarrow\infty} \frac{1}{T} \int_{-\frac{T}{2}}^{\frac{T}{2}} A^2 \cos^2(2\pi f_0 t + \phi) dt \nonumber\\
                              &= \lim_{T\rightarrow\infty} \frac{1}{T} \frac{A}{2}T + \lim_{T\rightarrow\infty} \frac{A}{2}\int_{-\frac{T}{2}}^{\frac{T}{2}}\cos(4\pi f_0 t + 2\phi) dt  \nonumber\\
                              &= \frac{A}{2} + \lim_{T\rightarrow\infty} \eval{\frac{A}{2}\frac{1}{4\pi f_0}\sin(4\pi f_0 t + 2\phi)}_{\frac{T}{2}}^{-\frac{T}{2}} = \frac{A^2}{2} \nonumber
                    \end{align}
                }
                \item {Valore Efficace:
                    \[
                        x_{eff} = \sqrt{P_{x}} = \sqrt{\frac{A^2}{2}} =\frac{|A^2|}{\sqrt{2}} 
                    \]
                }
                \item {Valore Medio:
                    \begin{align}
                        x_{m} &= \lim_{T\rightarrow\infty} \frac{1}{T} \int_{-\frac{T}{2}}^{\frac{T}{2}}  x_{(t)} \,dt =\lim_{T\rightarrow\infty} \frac{1}{T} \int_{-\frac{T}{2}}^{\frac{T}{2}}\cos(2\pi f_0 t+\phi) dt \nonumber \\
                              &= \lim_{T\rightarrow\infty} \frac{1}{T}  \eval{\frac{A}{2}\frac{1}{2\pi f_0}\sin(2\pi f_0 t + \phi)}_{\frac{T}{2}}^{-\frac{T}{2}} = 0 \nonumber
                    \end{align}
                }
            \end{itemize}

            %Richimo per il label del formulario $(1)$ e $(2)$ \ref{Trigonometria} 
        \pagebreak
        \subsubsection{Gradino}
        $U_{(t)} = x_{(t)} = 
            \begin{cases}
                1 & t > 0 \\
                0 & t \leq 0  
            \end{cases}
        $
        \begin{figure}[H]
            \centering
            \includegraphics[width=4cm]{media/uwu.png}
            \caption{Segnale gradino}
            \label{fig:segnale gradino}
            
        \end{figure}        

        \begin{itemize}
            \item {Energia:
                \[
                    E_{x} = \int_{-\infty}^{\infty} |x_{(t)}|^2 \ dt = \int_{-\infty}^{\infty} 1\ dt = \infty 
                \]
            }
            \item {Potenza Media:
                \[
                    P_{x} =\lim_{T\rightarrow\infty}  \frac{1}{T} \int_{-\frac{T}{2}}^{\frac{T}{2}}  |U_{(t)}|^2 \,dt =\lim_{T\rightarrow\infty} \frac{1}{T} \int_{-\frac{T}{2}}^{\frac{T}{2}} 1\ dt = \lim_{T\rightarrow\infty} \frac{1}{T} \frac{T}{2} = \frac{1}{2}
                \]
            }
            \item {Valore Efficace:
                \[
                    x_{eff} = \sqrt{P_{x}} = \frac{1}{\sqrt{2}} 
                \]
            }
            \item {Valore Medio:
                    \[x_{m} = \lim_{T\rightarrow\infty} \frac{1}{T} \int_{-\frac{T}{2}}^{\frac{T}{2}}  x_{(t)} \,dt =\lim_{T\rightarrow\infty} \frac{1}{T} \int_{-\frac{T}{2}}^{\frac{T}{2}} 1\ dt = \lim_{T\rightarrow\infty} \frac{1}{T} \frac{T}{2} = \frac{1}{2} \]
            }
        \end{itemize}
        
        \subsubsection{Rettangolo}
        $x_{(t)} = A\hspace{0.1cm}rect\left(\frac{t}{T}\right) =
            \begin{cases}
                A & -\frac{t}{T}\leq t\leq \frac{t}{T}\\
                0 & Altrove 
            \end{cases}
        $
        \begin{figure}[H]
            \centering
            \includegraphics[width=4cm]{media/uwu.png}
            \caption{Segnale rettangolo}
            \label{fig:segnale rettangolo}
        \end{figure}        
        \begin{itemize}
            \item {Energia:
                \[
                    E_{x} = \int_{-\infty}^{\infty} |x_{(t)}|^2 \ dt = \int_{-\frac{T}{2}}^{\frac{T}{2}} A^2 \hspace{0.1cm}rect^2\left(\frac{t}{T}\right)\ dt = A^2 \int_{-\frac{T}{2}}^{\frac{T}{2}}  1\ dt = A^2 T 
                \]
            }
            \item {Potenza Media:
                $T < T_0$ se non fosse cosí avrei una costante
                \begin{align}
                    P_{x} =\lim_{T\rightarrow\infty}  \frac{1}{T_0} \int_{-\frac{T_0}{2}}^{\frac{T_0}{2}}  |x_{(t)}|^2 \,dt & =\lim_{T\rightarrow\infty} \frac{1}{T_0}\int_{-\frac{T_0}{2}}^{\frac{T_0}{2}} A^2 \hspace{0.1cm}rect^2\left(\frac{t}{T}\right)\ dt \nonumber \\
                         & =\lim_{T\rightarrow\infty} \frac{1}{T_0} A^2 T = 0 \nonumber
                \end{align}
            }
            \item {Valore Efficace:
                \[
                    x_{eff} = \sqrt{P_{x}} = 0 
                \]
            }
            \item {Valore Medio:
                    \begin{align}
                        x_{m} = \lim_{T\rightarrow\infty} \frac{1}{T} \int_{-\frac{T}{2}}^{\frac{T}{2}}  x_{(t)} \,dt & = \lim_{T\rightarrow\infty} \frac{1}{T_0}\int_{-\frac{T_0}{2}}^{\frac{T_0}{2}} A\hspace{0.1cm}rect\left(\frac{t}{T}\right)\ dt \nonumber \\
                        & =\lim_{T\rightarrow\infty} \frac{1}{T_0} A T = 0 \nonumber
                    \end{align}
            }
        \end{itemize}
        
        
        \subsubsection{Esponenziale unilatera}
        $x_{(t)} = e^{-t}U_{(t)}$
        \begin{figure}[H]
            \centering
            \includegraphics[width=4cm]{media/uwu.png}
            \caption{Segnale esponenziale unilatera}
            \label{fig:segnale esponenziale unilatera}
        \end{figure}        
        \begin{itemize}
            \item {Energia:
                \[
                    E_{x} = \int_{-\infty}^{\infty} |x_{(t)}|^2 \ dt = \int_{0}^{\infty} e^{-2t}\ dt = \eval*{\frac{1}{2} e^{-2t}}_{0}^{\infty} = \frac{1}{2} 
                \]
            }
            \item {Potenza Media:
                \begin{align}
                    P_{x} & =\lim_{T\rightarrow\infty}  \frac{1}{T} \int_{-\frac{T}{2}}^{\frac{T}{2}}  |e^{-t}U_{(t)}|^2 \,dt =\lim_{T\rightarrow\infty} \frac{1}{T} \int_{0}^{\frac{T}{2}} e^{-2t}\ dt \nonumber \\
                          & = \lim_{T\rightarrow\infty} \frac{1}{T} \eval*{\left(-\frac{1}{2}\right) e^{-2t}}_{0}^{\frac{T}{2}} =\lim_{T\rightarrow\infty}-\frac{1}{2T} e^{-2\frac{T}{2}} + \lim_{T\rightarrow\infty} \frac{1}{2T} = 0 \nonumber 
                \end{align}
            }
            \item {Valore Efficace:
                \[
                    x_{eff} = \sqrt{P_{x}} = 0 
                \]
            }
            \item {Valore Medio:
                    \begin{align}
                        x_{m} & = \lim_{T\rightarrow\infty} \frac{1}{T} \int_{-\frac{T}{2}}^{\frac{T}{2}}  x_{(t)} \,dt =\lim_{T\rightarrow\infty} \frac{1}{T} \int_{-\frac{T}{2}}^{\frac{T}{2}} e^{-t}U_{(t)}\,dt = \lim_{T\rightarrow\infty} \frac{1}{T} \int_{0}^{\frac{T}{2}} e^{-t}\,dt \nonumber \\
                              & = \lim_{T\rightarrow\infty} \frac{1}{T} \eval*{(-1) e^{-t}}_{0}^{\frac{T}{2}} =  \lim_{T\rightarrow\infty}-\frac{1}{T} e^{-\frac{T}{2}} + \lim_{T\rightarrow\infty} \frac{1}{T} = 0 \nonumber
                    \end{align}
            }
        \end{itemize}
        
        \pagebreak
        \subsubsection{Esponenziale bilatera}
        $x_{(t)} = e^{-|t|}$
        \begin{figure}[H]
            \centering
            \includegraphics[width=4cm]{media/uwu.png}
            \caption{Segnale esponenziale bilatera}
            \label{fig:segnale esponenziale bilatera}
        \end{figure}        
        \begin{itemize}
            \item {Energia:
                \[
                    E_{x} = \int_{-\infty}^{\infty} |x_{(t)}|^2 \ dt =2 \int_{0}^{\infty} e^{-2t}\ dt = \eval*{2 \left(-\frac{1}{2}\right) e^{-2t}}_{0}^{\infty} = 1 
                \]
            }
            \item {Potenza Media:
                \begin{align}
                    P_{x} & =\lim_{T\rightarrow\infty}  \frac{1}{T} \int_{-\frac{T}{2}}^{\frac{T}{2}}  |e^{-t}U_{(t)}|^2 \,dt =\lim_{T\rightarrow\infty} \frac{2}{T} \int_{0}^{\frac{T}{2}} e^{-2t}\ dt \nonumber \\
                          & = \lim_{T\rightarrow\infty} \frac{1}{T} \eval*{e^{-2t}}_{0}^{\frac{T}{2}} =\lim_{T\rightarrow\infty}-\frac{1}{T} e^{-2\frac{T}{2}} + \lim_{T\rightarrow\infty} \frac{1}{T} = 0 \nonumber 
                \end{align}
            }
            \item {Valore Efficace:
                \[
                    x_{eff} = \sqrt{P_{x}} = 0 
                \]
            }
            \item {Valore Medio:
                    \begin{align}
                        x_{m} & = \lim_{T\rightarrow\infty} \frac{1}{T} \int_{-\frac{T}{2}}^{\frac{T}{2}}  x_{(t)} \,dt =\lim_{T\rightarrow\infty} \frac{1}{T} \int_{-\frac{T}{2}}^{\frac{T}{2}} e^{-t}U_{(t)}\,dt = \lim_{T\rightarrow\infty} \frac{1}{T} 2\int_{0}^{\frac{T}{2}} e^{-t}\,dt \nonumber \\
                              & = \lim_{T\rightarrow\infty} \frac{1}{T} \eval*{(-2) e^{-t}}_{0}^{\frac{T}{2}} =  \lim_{T\rightarrow\infty}-\frac{2}{T} e^{-\frac{T}{2}} + \lim_{T\rightarrow\infty} \frac{2}{T} = 0 \nonumber
                    \end{align}
            }
        \end{itemize}        

        \subsubsection{segno $\mathbf{sgn(x_{(t)})}$}
        $x_{(t)} = sgn(t) =
            \begin{cases}
                -1 & t < 0\\
                1  & t>0 
            \end{cases}
        $
        \begin{figure}[H]
            \centering
            \includegraphics[width=4cm]{media/uwu.png}
            \caption{Segnale sgn(x)}
            \label{fig:segnale sgn(x)}
        \end{figure}
        \begin{itemize}
            \item {Energia:
                \[
                    E_{x} = \int_{-\infty}^{\infty} |x_{(t)}|^2 \ dt = \int_{-\infty}^{\infty} sgn^2(t)\ dt = \int_{-\infty}^{\infty} 1\ dt =\infty 
                \]
            }
            \item {Potenza Media:
                \[
                    P_{x} =\lim_{T\rightarrow\infty}  \frac{1}{T} \int_{-\frac{T}{2}}^{\frac{T}{2}}  |x_{(t)}|^2 \,dt =\lim_{T\rightarrow\infty} \frac{1}{T} \int_{-\frac{T}{2}}^{\frac{T}{2}} sgn^2{t}\ dt = \lim_{T\rightarrow\infty} \frac{1}{T} T = 1
                \]
            }
            \item {Valore Efficace:
                \[
                    x_{eff} = \sqrt{P_{x}} = 1 
                \]
            }
            \item {Valore Medio:
                \begin{align}
                    x_{m} & = \lim_{T\rightarrow\infty} \frac{1}{T} \int_{-\frac{T}{2}}^{\frac{T}{2}}  x_{(t)} \,dt =\lim_{T\rightarrow\infty} \frac{1}{T} \int_{-\frac{T}{2}}^{\frac{T}{2}} sgn(t)\ dt \nonumber \\
                          & = \lim_{T\rightarrow\infty} \frac{1}{T} \left[\int_{-\frac{T}{2}}^{0}  1\,dt + \int_{0}^{\frac{T}{2}}  1\,dt\right] = \lim_{T\rightarrow\infty} \frac{1}{T} \left(-\frac{T}{2}+\frac{T}{2}\right) = 0 \nonumber
                \end{align}        
            }
        \end{itemize}
        
