\section{Trasformata Serie Di Fourier}
    \subsection{Segnale Periodico}
        Si definisce segnale periodico un segnale tale che:
        \[
            x_{(t)} = x_{(t-kT_0)}    
        \]
        \[
            T_0=Periodo\ \ \ f_0\triangleq \frac{1}{T_0} =Frequenza
        \]
    \subsection{Trasformata Serie Di Fourier}
        Ogni segnale periodico di periodo $T_0$ che soddifa le condizioni di Dirichlet e la sua $E_x < \infty (C.S.)$ puó essere scritto come la somma di 
        infinite sinusoidi di frequenze multiple di $f_0 = \frac{1}{T_0}$
        \begin{itemize}
            \item{Equazione di Sintesi - Antitrasformata(ATSF)\label{ATSF}
                \[
                    x_{(t)} = \sum_{k = -\infty}^{\infty} X_{k} e^{j2\pi kf_0t} \hspace{1cm} X_{k}\in \mathbb{C},\ f_0 = \frac{1}{T_0} 
                \]
                Se lo sviluppassimo sarebbe composto da:
                \[
                   x_{(t)} =\ldots  + X_{-1} e^{j2\pi (-1)f_0t} + X_{0} + X_{1} e^{j2\pi (1) f_0t} + \ldots 
                \]
                $X_0$ corrisponde al Valore medio \ref{Valore medio} del segnale, inoltre le componenti $X_k$ prendono il nome di armoniche alla frequenza $f$ corrispondente
            }
            \item{Equazione di Analisi - Trasformata(TSF)\label{TSF}
                \[
                    X_k =\frac{1}{T_0}\int_{-\frac{1}{T_0}}^{\frac{1}{T_0}} x_{(t)} e^{-j2\pi kf_0t} dt
                \]
            } 
        \end{itemize}
        La TSF gode della biunivocitá: $\forall x_{(t)} \exists! X_k$:
        \begin{align}
            x_{(t)} & \rightleftharpoons X_k  \nonumber\\
            Segnale\ Analogico\ Periodico &\rightleftharpoons Sequenza\ Complessa \nonumber
        \end{align}
        \subsubsection{Rappresentazione di $X_k$}
            Essendo $X_k$ un numero complesso puó essere rappresentato in forma polare: 
            \[
                X_k = |X_k|e^{\angle X_k}  
            \]
            Si possono rappresentare il modulo (Ampiezza) e la fase tramite grafici che prendono il nome di spettri:
            \begin{figure}[H]
                \centering
                \subfloat[Spettro di Ampiezza]{\includegraphics[width=0.4\textwidth]{media/uwu.png}\label{fig:Spettro di Ampiezza}}
                \hfill
                \subfloat[Spettro di Fase]{\includegraphics[width=0.4\textwidth]{media/uwu.png}\label{fig:Spettro di Fase}}
                \caption{Spettro del segnale}
            \end{figure}
            lo spettro di Ampiezza gode della \textbf{simmetria pari} rispetto alle ascisse quindi é \textbf{sempre positivo}, mentre lo spettro di fase della \textbf{simmetria dispari}.
        \subsection{Calcolo dei coefficenti $X_k$ per segnali noti}
            \subsubsection{$A\cos(2\pi f_0 t)$}
                $x_{(t)}=A\cos(2\pi f_0 t)$
                \begin{align}
                    ATSF[x_{(t)}] & = ATSF[A\cos(2\pi f_0 t)] \nonumber \\
                        & = ATSF[\frac{A}{2} (e^{j2\pi kf_0t} + e^{-j2\pi kf_0t})] \nonumber 
                \end{align}
                Utilizzando la composizione dei coefficenti $X_k$:
                \begin{align}
                    x_{(t)} & =\ldots  + X_{-1} e^{j2\pi (-1)f_0t} + X_{0} + X_{1} e^{-j2\pi (1) f_0t} + \ldots \nonumber\\
                    Abbiamo:& \nonumber 
                \end{align}
                        \[X_{-1} = \frac{A}{2}\hspace{.3cm} X_{0} = 0\hspace{.3cm} X_{1} = \frac{A}{2}\] 
                Possiamo tracciare lo spettro del segnale:
                \begin{figure}[H]
                    \centering
                    \subfloat[Spettro di Ampiezza]{\includegraphics[width=0.4\textwidth]{media/uwu.png}\label{fig:sa TSF coseno}}
                    \hfill
                    \subfloat[Spettro di Fase]{\includegraphics[width=0.4\textwidth]{media/uwu.png}\label{fig:sf TSF coseno}}
                    \caption{Spettro TSF del coseno}
                \end{figure}
            \subsubsection{$A\sin(2\pi f_0 t)$}
                $x_{(t)}=A\sin(2\pi f_0 t)$
                \begin{align}
                    ATSF[x_{(t)}] & = ATSF[A\sin(2\pi f_0 t)] \nonumber \\
                        & = ATSF[\frac{A}{2} (e^{j2\pi kf_0t} - e^{-j2\pi kf_0t})] \nonumber 
                \end{align}
                Utilizzando la composizione dei coefficenti $X_k$:
                \begin{align}
                    x_{(t)} & =\ldots  + X_{-1} e^{j2\pi (-1)f_0t} - X_{0} + X_{1} e^{-j2\pi (1) f_0t} + \ldots \nonumber\\
                    Abbiamo:& \nonumber 
                \end{align}
                        \[X_{-1} = -\frac{A}{2j}\hspace{.3cm} X_{0} = 0\hspace{.3cm} X_{1} = \frac{A}{2j}\] 
                \[
                    |X_k|= 
                    \begin{cases}
                            |\frac{A}{2j}| = \frac{A}{2} \hspace{0.5cm} & k= 1\\
                            |-\frac{A}{2j}| = \frac{A}{2} \hspace{0.5cm} & k= -1\\
                            0 \hspace{0.5cm} & altrove  \\
                    \end{cases}
                    \hspace{0.5cm}
                    \angle X_k= 
                    \begin{cases}
                        \angle \frac{A}{2j} = -\frac{\pi}{2} \hspace{0.5cm} & k= 1\\
                        \angle |-\frac{A}{2j}| = \frac{\pi}{2} \hspace{0.5cm} & k= -1\\
                        0 \hspace{0.5cm} & altrove  \\
                    \end{cases}
                    \]
                Possiamo tracciare lo spettro del segnale:
                \begin{figure}[H]
                    \centering
                    \subfloat[Spettro di Ampiezza]{\includegraphics[width=0.4\textwidth]{media/uwu.png}\label{fig:sa TSF seno}}
                    \hfill
                    \subfloat[Spettro di Fase]{\includegraphics[width=0.4\textwidth]{media/uwu.png}\label{fig:sf TSF seno}}
                    \caption{Spettro TSF del seno}
                \end{figure}
            \subsubsection{Treno di rect}
                $x_R=A\hspace{0.1cm}rect\left(\frac{t}{T}\right)\rightarrow$ Segnale periodico $\rightarrow x_{(t)} = \sum_{-\infty}^{\infty} x_R (t-nT_0)$\\
                $T_0 = periodo$, $T = durata \rightarrow T < T_0$, se cosi non fosse avremmo una costante
                \begin{figure}[H]
                    \centering
                    \includegraphics[width=4cm]{media/uwu.png}
                    \caption{Treno di $A\hspace{0.1cm}rect\left(\frac{t}{T}\right)$}
                    \label{fig:treno di rect}
                \end{figure}                
                $\rightarrow$ Si nota come cambiare il periodi delle funzioni possiamo renderle da aperiodiche a periodiche e viceversa
                \begin{align}
                    ATSF[x_{(t)}] & = ATSF[\sum_{-\infty}^{\infty} x_R (t-nT_0)] \nonumber \\
                        & = \frac{1}{T_0} \int_{-\frac{T_0}{2}}^{\frac{T_0}{2}} x_{(t)} e^{-j2\pi kf_0t} dt = \frac{1}{T_0} \int_{-\frac{T}{2}}^{\frac{T}{2}} x_{(t)} e^{-j2\pi kf_0t} dt \nonumber \\
                        & = \frac{A}{T_0} \int_{-\frac{T}{2}}^{\frac{T}{2}} e^{-j2\pi kf_0t} dt = \eval{\frac{A}{T_0} \frac{1}{j2\pi kf_0} e^{-j2\pi kf_0t}}_{-\frac{T}{2}}^{\frac{T}{2}} \nonumber \\
                        & = \frac{A}{T_0} \frac{e^{-j\pi kf_0T} - e^{j\pi kf_0T}}{j2\pi kf_0} = \frac{A}{T_0} \frac{e^{j\pi kf_0T}-e^{-j\pi kf_0T}}{-j2\pi kf_0} \nonumber \\
                        & = \frac{A\color{purple}{T}}{T_0} \frac{e^{j\pi kf_0T}-e^{-j\pi kf_0T}}{-j2\pi kf_0\color{purple}{T}} = Af_0T sinc(kf_0T) \nonumber 
                \end{align}
                \begin{figure}[H]
                    \centering
                    \subfloat[Spettro di Ampiezza]{\includegraphics[width=0.4\textwidth]{media/uwu.png}\label{fig:sa TSF treno di rect}}
                    \hfill
                    \subfloat[Spettro di Fase]{\includegraphics[width=0.4\textwidth]{media/uwu.png}\label{fig:sf TSF treno di rect}}
                    \caption{Spettro TSF del treno di rect}
                \end{figure}
                Si possono anche unire i due spettri per ottenere: 
                \begin{figure}[H]
                    \centering
                    \includegraphics[width=4cm]{media/uwu.png}
                    \caption{Spettro treno di $A\hspace{0.1cm}rect\left(\frac{t}{T}\right)$}
                    \label{fig:Spettro treno di rect}
                \end{figure}  
            Ora appizza matlab e fa esempi di un segnale e uno di riostruzione dello stesso(script di matlab presenti nel teams):
            \begin{itemize}
                \item Se un segnale varia molto rapidamente nel tempo ha componenti frequenziali piú alte $\rightarrow$ copre piú spettro(espansione spettrale) $T_0 \uparrow$
                \item Se un segnale varia molto lentamente copre le basse fraquenze $ T_0 \downarrow$
            \end{itemize}
            Se non ho abbastanza passi K non posso campionare le alte frequenze e quindi non faccio ne un analisi completa del segnale né riesco a ricostruire perfettametne il segnale  












