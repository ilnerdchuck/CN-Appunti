\section{Introduzione Ai Segnali}
    \begin{itemize}
        \item {
            Deterministici: Segnale rappresentabile con funzioni analitiche e noto $\forall t$
        }
        \item {
            Aleatori: Segnale rappresentabile tramite statistiche 
        }
    \end{itemize}
    \subsection{Classificazione di segnale in base alla continuità dei domini}
        \begin{itemize}
            \item {Dominio del tempo:
                    \begin{itemize}
                        \item{Segnale tempo continuo: $t \in \mathbb{R}$ assume con conitinuità tutti i valori contenuti all'interno di un intervallo}
                        \item {Segnale a tempo discreto: $t = \{ nT \} n \in \mathbb{Z} \ T=$periodo di campionamento, la variabile temoporale assume solo valori discreti}
                    \end{itemize}
                    \begin{figure}[h]
                        \centering
                        \includegraphics[width=4cm]{media/uwu.png}
                        \caption{\color{purple}{tempo continuo}, \color{blue}{tempo discreto}}
                        \label{fig:Dominio del tempo}
                    \end{figure}
            }
            \item {Dominio dell'ampiezza (spazio):
                    \begin{itemize}
                        \item{Segnale ad ampiezza continua: $x_{(t)}\ continua$, la grandezza fisica del segnale assume con continuità tutti i valori all'interno di un intervallo}
                        \item {Segnale ad ampiezza discreta: $x_{(t)}\ discreta$,se restringo l'intervallo posso renderla continua, la grandezza fisica puó assumere solo valori discreti}
                    \end{itemize}
                    \begin{figure}[h]
                        \centering
                        \includegraphics[width=4cm]{media/uwu.png}
                        \caption{\color{purple}{ampiezza continua}, \color{blue}{ampiezza discreta}}
                        \label{fig:dominio dell'ampiezza}
                    \end{figure}
            }
        \end{itemize}
        \begin{table}[h]
            \centering
            \begin{tabular}{c|cccc}
            Segnale   & \multicolumn{1}{c|}{Cotinuo}     & Discreto          & $t$ &  \\ \cline{1-4}
            Continua  & \multicolumn{1}{c|}{Analogico}   & Sequenza/Digitale &       &  \\ \cline{1-3}
            Discreta  & \multicolumn{1}{c|}{Quantizzato} & Binario           &       &  \\
            $x_{(t)}$ &                                  &                   &       & 
            \end{tabular}
        \end{table}
