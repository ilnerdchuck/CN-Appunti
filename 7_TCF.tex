\section{Trasformata Continua Di Fourier}
    \subsection{Segnali Aperiodici}
        Nel caso di segnali come $x_{(t)}=rect\left(\frac{t}{T}\right)$ non posso usare la $TSF$ posso peró scrivere:
        \[
            x_{(t)} = \lim_{T_0\rightarrow\infty} x_p(t) ,\ x_p(t) = \sum_{n = -\infty}^{\infty} x_{(t-nT_0)} 
        \]
        Passiamo da un analisi a frequenze discrete ad un analisi su tutto lo spettro delle frequenze
        \begin{figure}[H]
            \centering
            \subfloat[Spettro di Ampiezza TSF]{\includegraphics[width=0.4\textwidth]{media/uwu.png}\label{fig:Spettro di Ampiezza TSF}}
            \hfill
            \subfloat[Spettro di Ampiezza TCF]{\includegraphics[width=0.4\textwidth]{media/uwu.png}\label{fig:Spettro di Ampiezza TCF}}
        \end{figure}

    \subsection{Equazioni di Analisi e Sintesi}
        \begin{figure}[H]
            \centering
            \includegraphics[width=4cm]{media/uwu.png}
            \caption{Insieme dei segnali per tcf}
            \label{fig:segnali aperiodici tcf}
        \end{figure}
        {\subsubsection{Equazione di Analisi}
            \[X_{(f)} = \int_{-\infty}^{\infty} x_{(t)} e^{-j2\pi ft} dt\hspace{0.3cm} Equazione\ di\ analisi \]
            
        \subsubsection{Equazione di Sintesi}
            \[x_{(t)} = \int_{-\infty}^{\infty} X_{(f)} e^{j2\pi ft} df\hspace{0.3cm} Equazione\ di\ sintesi \]
        }
        La $TCF$ gode della biunivocitá
        \begin{align}
            x_{(t)} \rightleftharpoons  X_{(f)}\nonumber \hspace{0.3cm} X_{(f)} \in \mathbb{C}
        \end{align}
        Essendo $X_k$ un numero complesso puó essere rappresentato in forma polare: 
        \[
            X_{(f)} = |X_{(f)}|e^{\angle X_{(f)}}  
        \]
        Si possono rappresentare il modulo (Ampiezza) e la fase tramite grafici che prendono il nome di spettri:
        \begin{figure}[H]
            \centering
            \subfloat[Spettro di Ampiezza]{\includegraphics[width=0.4\textwidth]{media/uwu.png}\label{fig:Spettro di Ampiezza TCF}}
            \hfill
            \subfloat[Spettro di Fase]{\includegraphics[width=0.4\textwidth]{media/uwu.png}\label{fig:Spettro di Fase TCF2}}
            \caption{Spettro del segnale TCF}
        \end{figure}
        lo spettro di Ampiezza gode della \textbf{simmetria pari} rispetto alle ascisse quindi é \textbf{sempre positivo e continuo}, mentre lo spettro di fase della \textbf{simmetria dispari}, 
        questa propietá é chiamata \textbf{Simmetria Hermitiana}

        \subsubsection{TCF di una $Arect\left(\frac{t}{T}\right)$}
            $x_{(t)} = A\hspace{0.1cm}rect\left(\frac{t}{T}\right)$
            \begin{figure}[H]
                \centering
                \includegraphics[width=4cm]{media/uwu.png}
                \caption{$A\hspace{0.1cm}rect\left(\frac{t}{T}\right)$}
                \label{fig:grafo rect nella tcf}
            \end{figure}
            $X_{(f)} = ? :$
            \begin{align}
                X_{(f)} & = \int_{-\infty}^{\infty} x_{(t)} e^{-j2\pi ft} dt \nonumber = \int_{-\frac{T}{2}}^{\frac{T}{2}} A\hspace{0.1cm}rect\left(\frac{t}{T}\right) e^{-j2\pi ft} dt \nonumber\\ 
                        & = A \int_{-\frac{T}{2}}^{\frac{T}{2}} e^{-j2\pi ft} dt = -\frac{A}{j2\pi f} \eval{e^{-j2\pi ft}}_{-\frac{T}{2}}^{\frac{T}{2}} =-\frac{A}{j2\pi f} \left(e^{-j\pi fT} - e^{j\pi fT}\right) \nonumber\\
                        & = \frac{A\color{purple}{T}}{\pi f} \left(\frac{e^{j\pi fT} - e^{-j\pi fT}}{2j}\right) = \frac{A{\color{purple}T}\sin(\pi fT)}{\pi f\color{purple}{T}} = AT sinc(fT) =  X_{(f)} \nonumber 
            \end{align}
            \[
                A\hspace{0.1cm}rect\left(\frac{t}{T}\right) \rightleftharpoons AT sinc(fT)
            \]
            La sinc si annulla in $\frac{k}{T},\ k\in \mathbb{Z}$. Notiamo anche come la funzione di partenza sia reale e pari la TCF rispetti \ref{Parita}(si?):
            \begin{figure}[H]
                \centering
                \subfloat[Spettro di Ampiezza TCF $rect$]{\includegraphics[width=0.4\textwidth]{media/uwu.png}\label{fig:Spettro di Ampiezza TCF rect}}
                \hfill
                \subfloat[Spettro di Fase TCF $rect$]{\includegraphics[width=0.4\textwidth]{media/uwu.png}\label{fig:Spettro di Fase TCF rect}}
            \end{figure}
            
    \subsection{Propietá}
        Come per la TSF vale che al variare del periodo della funzione $T$:
            \begin{itemize}
                \item Se $T\uparrow$ aumenta $ \rightarrow f\downarrow$ diminuisce e si stringe lo spettro  
                \item Se $T\downarrow$ diminuisce $ \rightarrow f\uparrow$ aumenta e si allarga lo spettro  
            \end{itemize}
        Inoltre come si puó evincere dal successivo Teorema della Dualitá \ref{Dualita}:
            \begin{itemize}
                \item Una funzione limitata (finita) nel tempo ha uno spettro nella frequenza illimitato $\rightarrow$ sono i segnali fisici   
                \item Una funzione illimitata nel tempo ha uno spettro nella frequenza limitato (finito)
            \end{itemize}
        \subsubsection{Simmetria hermitiana}\label{Simmetria Hermitiana}
            \begin{align}
                Ip&: x_{(t)}\ reale \nonumber \\
                Th&: X_{(f)}\ hermitiana \nonumber \\ 
                X_{(-f)} &= X_{(f)}^{*} \rightarrow
                    \begin{cases}
                        |X_{(f)}| = |X_{(-f)}| \hspace{0.3cm} & Simmetria\ Pari \\
                        \angle X_{(-f)} = -\angle X_{(f)}\hspace{0.3cm} & Simmetria\ Dispari
                    \end{cases} \nonumber
            \end{align}
        \subsubsection{Paritá}\label{Parita}
            \begin{align}
                Ip&: x_{(t)}\ reale\ e\ pari  \nonumber \\
                Th&: X_{(f)}\ reale\ e\ pari \nonumber  
            \end{align}
        \subsubsection{Disparitá}\label{Disparita}
            \begin{align}
                Ip&: x_{(t)}\ reale\ e\ dispari  \nonumber \\
                Th&: X_{(f)}\ immaginaria\ e\ dispari \nonumber 
            \end{align}
    \subsection{Teoremi relativi alla TCF}
        \subsubsection{Linearitá}\label{Linearita}
            $Ip: x_{(t)} = \alpha x_{1(t)} + \beta x_{2(t)}$\\        
            $Th: X_{(f)} = \alpha X_{1(f)} + \beta X_{2(f)}$\\ 
            Dimostrazione:
            \begin{align}
                X_{(f)} & = \int_{-\infty}^{\infty} (\alpha x_{1(t)} + \beta x_{2(t)}) e^{-j2\pi ft} dt \nonumber \\
                        & = \alpha \int_{-\infty}^{\infty} x_{1(t)} e^{-j2\pi ft} dt + \beta \int_{-\infty}^{\infty}  x_{2(t)} e^{-j2\pi ft} dt  \nonumber \\
                        & = \alpha X_{1(f)} + \beta X_{2(f)} \nonumber
            \end{align}

            Esempio:
                {
                    esempio del martorella rect su rect
                }

        \subsubsection{Dualitá}\label{Dualita}
            $Ip: x_{(t)} \rightleftharpoons^{TCF} X_{(f)}$\\        
            $Th: X_{(t)} \rightleftharpoons^{TCF} x_{(-f)}$ 
            Dimostrazione:
            \begin{align}
                X_{(f)} & = \int_{-\infty}^{\infty}  e^{-j2\pi ft} dt \nonumber \\
                        & = finisci\nonumber \\
                        & =  \nonumber
            \end{align}

        \subsubsection{Ritardo}\label{Ritardo}
            $Ip: x_{(t)} = $\\        
            $Th: X_{(f)} = $ 

        \subsubsection{Derivazione}\label{Derivazione}
            $Ip: x_{(t)} = $\\        
            $Th: X_{(f)} = $ 
            
        \subsubsection{Integrazione}\label{Integrazione}
            $Ip: x_{(t)} = $\\        
            $Th: X_{(f)} = $ 
        
        \subsubsection{Derivazione in Frequenza}\label{Derivazione in Frequenza}
            $Ip: x_{(t)} = $\\        
            $Th: X_{(f)} = $ 
            
        \subsubsection{Integrazione in Frequenza}\label{Integrazione in Frequenza}
            $Ip: x_{(t)} = $\\        
            $Th: X_{(f)} = $ 
        
        \subsubsection{Convoluzione}\label{Convoluzione}
            $Ip: x_{(t)} = $\\        
            $Th: X_{(f)} = $ 
            
            Propietá della convoluzione:
            \begin{itemize}
                \item {1}
                \item {2}
                \item {3}
            \end{itemize}
        \subsubsection{Prodotto}\label{Prodotto}
    \subsection{Modulazione di Ampiezza}\label{Modulazione di Ampiezza}
        \subsubsection{Modulazione con $\cos(2\pi f_0t)$}\label{Modulazione con coseno}
            $Ip: x_{(t)} = $\\        
            $Th: X_{(f)} = $ 
            
        \subsubsection{Modulazione con $\sin(2\pi f_0t)$}\label{Modulazione con seno}
            $Ip: x_{(t)} = $\\        
            $Th: X_{(f)} = $ 
            
        \subsubsection{Modulazione con Esponenziale Complesso}\label{Modulazione con Esponenziale Complesso}
            $Ip: x_{(t)} = $\\        
            $Th: X_{(f)} = $ 
            
    tabellina sul procedimento di sintesi di un segnale 