\section{Teoria Dei Codici}
    \subsection{Introduzione}
        Ci concentriamo adesso sul trattamento dell'informazione per poterla trasmettere.
        I messaggi che trasmettiamo possono essere codificati per vari motivi:
        \begin{itemize}
            \item {
                    Compressione:$\begin{cases}
                        \text{Lossy: con perdita dell'informazione} \nonumber\\
                        \text{Lossless: minima perdita dell'informazione} \nonumber
                    \end{cases}$\\
                    Comprimere l'informazione in elimenando ridondanza e salvando spazio di memoria e banda.
                    
                }
            \item {
                    Crittografia: per nascondere il messaggio ad utenti in ascolto sul canale che non siano il destinatario.
            }
            \item {
                    Rivelazione o correzione di errore: vieen aggiunta ridondanza ad hoc per aumentare l'affidabilitá del messaggio trasmesso. 
                    Si utilizzano \href{https://en.wikipedia.org/wiki/Checksum}{checksum} o \href{https://en.wikipedia.org/wiki/Reed-Solomon_error_correction}{Reed-Solomon(RS)}
            }
        \end{itemize}

        \subsubsection{Esempio codici a blocco: codici a ripetizione}
            
        \subsubsection{Esempio codici a blocco: codici a controllo di paritá}

        \subsubsection{Esempio codici a blocco: codice ISBN}
